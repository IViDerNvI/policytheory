\subsection{政府工作报告}

2025年全国人民代表大会于3月5日至3月13日在北京召开。会议审议并通过了《政府工作报告》。

\subsubsection{2024年政府工作回顾}

\begin{itemize}
    \item 二十届三中全会胜利召开
    \item 庆祝中华人民共和国成立75周年
    \item 经济运行稳的态势持续巩固,进的步伐坚定有力
    \item 七方面工作内容
    \item 中国特色外交开创新局面
\end{itemize}

\paragraph{做好经济工作需要把握的五个关系} 统筹好有效市场和有为政府、总供给和总需求、培育新动能和更新旧动能、做优增量和盘活存量、提升质量和做大总量的关系。

\paragraph{全年经济运行态势} 前高,中低,后扬。

\paragraph{2024年政府工作七方面内容} 会议部署2024年政府工作七方面内容,包括因时因势加强和创新宏观调控,推动经济回升向好;坚定不移全面深化改革扩大开放,增强发展内生动力;大力推动创新驱动发展,促进产业结构优化升级;统筹城乡区域协调发展,优化经济布局;积极发展社会事业,增进民生福祉;持续加强生态环境保护,提升绿色低碳发展水平;加强政府建设和治理创新,保持社会和谐稳定。

\paragraph{外交新局面} 参加了上海合作组织峰会,金砖国家领导人会晤,亚太经合组织领导人非正式会议,二十国集团领导人峰会,东亚合作领导人系列会议等重大双边活动。成功举办中非合作论坛北京峰会,和平共处五项原则发表70周年纪念大会,中阿合作论坛部长级会议等。

\subsubsection{2025年经济社会发展总体要求和政策取向}

\paragraph{总体要求} 实施更加有为的宏观政策,扩大国内需求,推动科技创新和产业创新融合发展,稳住楼市股市,防范化解重点领域风险和外部冲击,稳定预期激发活力,推动经济持续回升向好,保持社会和谐稳定。

\paragraph{发展预期目标} 国内生产总值增长5\%左右;城镇调查失业率5.5\%以下;居民消费价格涨幅2\%左右;城镇新增就业1200万人以上;居民收入增长和经济增长同步;国际收支基本平衡;粮食产量1.4万亿斤左右;单位国内生产总值能耗降低3\%左右

\subsubsection{2025年政府工作任务}


\paragraph{提振内需,优化投资}
实施促消费专项行动,改善消费环境,推动居民增收。安排超长期特别国债支持消费品更新。加大基础设施和民间投资力度,完善专项债机制和项目审批流程。

\paragraph{发展新质生产力}
加快新兴产业和未来产业布局,推进制造业数字化转型,支持中小企业技术升级,壮大数字经济和智能产业。

\paragraph{强化科教人才}
实施教育强国三年行动,推动教育公平与质量提升。加强基础研究与核心技术攻关,完善科技成果转化机制,健全人才培养和引进机制。

\paragraph{深化重点改革}
加快建设全国统一大市场,规范涉企执法。推进财税金融体制改革,提升财政资金使用效率,扩大金融服务实体经济能力。

\paragraph{推进高水平对外开放}
稳定外贸,优化服务贸易和跨境电商发展。保障外资企业国民待遇,拓展自由贸易试验区改革。深化“一带一路”建设,推进区域与全球经贸合作。

\paragraph{防范重点风险}
稳房地产、化地方债、防金融风险。实施城中村改造,推进房企风险处置。优化隐性债务置换机制,推动中小金融机构转型。

\paragraph{乡村振兴与三农工作}
稳定粮食生产,推进高标准农田建设。巩固脱贫成果,发展农村产业与新型集体经济,优化乡村基础设施和生态环境。

\paragraph{新型城镇化与区域协调发展}
推进农业转移人口市民化,补齐县域基础设施短板。实施区域重大战略,推动产业梯度转移与都市圈协调发展。

\paragraph{绿色转型发展}
深入打好污染防治攻坚战,完善生态保护机制。推进碳达峰碳中和试点,发展绿色低碳技术和循环经济,扩大可再生能源规模。

\paragraph{保障民生与社会治理}
加强就业优先政策,完善医疗、社保、托育和养老服务。强化公共安全、防灾减灾和基层治理,推进法治政府和数字政府建设。

\paragraph{其他重点工作} 加强民族、宗教、侨务工作,推进民族团结和边疆发展;支持国防和军队现代化建设,强化军政军民团结。贯彻“一国两制”,支持港澳更好融入国家发展大局。推动两岸关系和平发展,反对“台独”,促进祖国统一。坚持独立自主和平外交政策,积极参与全球治理。

\paragraph{工作要求}
坚持实干导向,强化依法行政与服务意识,推进数字政府建设,整治形式主义。各级政府干部要提升能力、改进作风,确保党中央决策部署贯彻落实。



\subsection{中央城市工作会议}

7月14日至15日,中央城市工作会议在北京举行。

\subsubsection{近年城市工作取得的成就}

党的十八大以来,党中央深刻把握新形势下我国城市发展规律,坚持党对城市工作的全面领导,坚持人民城市人民建、人民城市为人民,坚持把城市作为有机生命体系统谋划,推动城市发展取得历史性成就,我国新型城镇化水平和城市发展能级、规划建设治理水平、宜业宜居水平、历史文化保护传承水平、生态环境质量大幅提升。

\subsubsection{总体要求}

当前和今后一个时期城市工作的总体要求是:

\begin{itemize}
    \item 坚持以习近平新时代中国特色社会主义思想为指导,深入贯彻党的二十大和二十届二中、三中全会精神,全面贯彻习近平总书记关于城市工作的重要论述。
    \item 坚持和加强\textbf{党的全面领导},认真践行人民城市理念。
    \item 坚持\textbf{稳中求进工作总基调},坚持因地制宜、分类指导。
    \item 以建设创新、宜居、美丽、韧性、文明、智慧的现代化人民城市为目标。
    \item 以推动城市高质量发展为\textbf{主题}。
    \item 以坚持城市内涵式发展为\textbf{主线}。
    \item 以\textbf{推进城市更新为重要抓手}。
    \item 大力推动\textbf{城市结构优化、动能转换、品质提升、绿色转型、文脉赓续、治理增效}。
    \item 牢牢守住城市安全底线,走出一条中国特色城市现代化新路子。
\end{itemize}

\subsubsection{主要内容}

\paragraph{城市工作总体形势和方向} 会议指出城市工作总体形势和方向

\begin{itemize}
    \item 我国城镇化正从\textbf{快速增长期转向稳定发展期,城市发展正从大规模增量扩张阶段转向存量提质增效为主的阶段}。
    \item 城市工作要深刻把握、主动适应形势变化,转变城市发展理念,更加注重\textbf{以人为本}。
    \item 转变城市发展方式,更加注重集约高效。
    \item 转变城市发展动力,更加注重特色发展。
    \item 转变城市工作重心,更加注重治理投入。
    \item 转变城市工作方法,更加注重统筹协调。
\end{itemize}

\paragraph{城市工作7个方面的重点任务} 会议部署城市工作七个方面的重点任务

\begin{itemize}
    \item 着力优化现代化城市体系。着眼于提高城市对人口和经济社会发展的综合承载能力,发展\textbf{组团式、网络化的现代化城市群和都市圈},分类推进\textbf{以县城为重要载体的城镇化建设,继续推进农业转移人口市民化},促进大中小城市和小城镇协调发展,促进城乡融合发展。
    \item 着力建设富有活力的创新城市。精心培育创新生态,在发展新质生产力上不断取得突破;依靠改革开放增强城市动能,高质量开展城市更新,充分\textbf{发挥城市在国内国际双循环中的枢纽作用}。
    \item 着力建设舒适便利的宜居城市。坚持人口、产业、城镇、交通一体规划,优化城市空间结构;加快构建\textbf{房地产发展新模式},稳步推进城中村和危旧房改造;大力发展\textbf{生活性服务业},提高公共服务水平,牢牢兜住民生底线。
    \item 着力建设绿色低碳的美丽城市。巩固生态环境治理成效,采取更有效措施解决城市空气治理、饮用水源地保护、新污染物治理等方面的问题,推动减污降碳扩绿协同增效,提升城市生物多样性。
    \item 着力建设安全可靠的韧性城市。推进城市基础设施生命线安全工程建设,加快老旧管线改造升级;\textbf{严格限制超高层建筑},全面提升房屋安全保障水平;强化城市自然灾害防治,统筹城市防洪体系和内涝治理;加强社会治安整体防控,切实维护城市公共安全。
    \item 着力建设崇德向善的文明城市。完善历史文化保护传承体系,完善城市风貌管理制度,保护城市独特的历史文脉、人文地理、自然景观;加强城市文化软实力建设,提高市民文明素质。
    \item 着力建设便捷高效的智慧城市。坚持党建引领,坚持依法治市,创新城市治理的理念、模式、手段,用好市民服务热线等机制,高效解决群众急难愁盼问题。
\end{itemize}

\paragraph{城市工作的重要原则} 会议强调城市工作要坚持的原则

\begin{itemize}
    \item 建设现代化人民城市,必须加强党对城市工作的全面领导。
    \item 要进一步健全领导体制和工作机制,增强城市政策协同性,强化各方面执行力。
    \item 要树立和践行正确政绩观,建立健全科学的城市发展评价体系,加强城市工作队伍素质和能力建设,激励广大党员干部干事创业、担当作为。
    \item 要坚持实事求是、求真务实,坚决反对形式主义、官僚主义。
\end{itemize}

\subsubsection{总结}

习近平总书记的重要讲话,科学回答了城市发展为了谁、依靠谁以及建设什么样的城市、怎样建设城市等重大理论和实践问题,为做好新时代新征程的城市工作提供了根本遵循,要认真学习领会,不折不扣抓好落实。

\begin{itemize}
    \item 深刻把握我国城市发展所处历史方位,以更加开阔的视野做好城市工作。
    \item 深刻把握建设现代化人民城市的目标定位,自觉践行以人民为中心的发展思想。
    \item 深刻把握城市内涵式发展的战略取向,更有针对性地提升城市发展质量。
    \item 深刻把握增强城市发展动力活力的内在要求,做好改革创新大文章。
    \item 深刻把握城市工作的系统性复杂性,着力提高落实各项任务部署的能力。
\end{itemize}
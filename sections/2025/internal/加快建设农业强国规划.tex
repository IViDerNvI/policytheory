\subsection{加快建设农业强国规划}

\subsubsection{总体要求}

\begin{itemize}
    \item 坚持以习近平新时代中国特色社会主义思想为指导,深入贯彻党的二十大和二十届二中、三中全会精神,全面贯彻习近平总书记关于“三农”工作的重要论述,完整准确全面贯彻新发展理念,加快构建新发展格局,着力推动高质量发展,立足国情农情、体现中国特色,保持战略定力、坚持久久为功。
      \item 坚持\textbf{农业农村优先}发展,运用“千万工程”经验\footnote{“千村示范、万村整治”工程:2003年6月,时任浙江省委书记的习近平同志在广泛深入调查研究基础上,立足浙江省情农情和发展阶段特征,准确把握经济社会发展规律和必然趋势,审时度势,高瞻远瞩,作出了实施“千万工程”的战略决策,提出从全省近4万个村庄中选择1万个左右的行政村进行全面整治,把其中1000个左右的中心村建成全面小康示范村。},健全推动乡村全面振兴长效机制,完善城乡融合发展体制机制,一体推进农业现代化和农村现代化,更高水平\textbf{守牢国家粮食安全底线和耕地保护红线}。
  \item 加快建设宜居宜业和美乡村,有力有效推进乡村全面振兴,加快建设供给保障强、科技装备强、经营体系强、产业韧性强、竞争能力强的农业强国,让广大农民过上更加美好的生活,为以中国式现代化全面推进强国建设、民族复兴伟业提供有力支撑。
\end{itemize}

工作中要做到:

\begin{itemize}
    \item 夯实产能、筑牢根基,把提高农业综合生产能力摆在更加突出的位置,紧紧围绕粮食和重要农产品供给保障短板,打牢设施基础、技术基础、装备基础、政策基础、制度基础,高水平把牢粮食安全主动权。
    \item 创新驱动、厚植动能,坚持创新在现代化建设全局中的核心地位,强化科技和改革双轮驱动,加大农业关键核心技术攻关力度,进一步全面深化农村改革,提高农业创新力、竞争力、全要素生产率,增强农业强国建设后劲。
    \item 绿色低碳、彰显底色,牢固树立绿水青山就是金山银山的理念,加快形成绿色低碳生产生活方式,让绿色循环、低碳发展成为农业强国的鲜明底色。
    \item 共建共享、富裕农民,注重保护小农户利益,以\textbf{促进农民持续增收、城乡收入差距持续缩小为重点},统筹推进产业发展、务工就业,完善强农惠农富农支持制度,牢牢守住不发生规模性返贫底线,促进城乡共同繁荣发展,让农业强国建设成果更多更公平惠及农民。
    \item 循序渐进、稳扎稳打,立足资源禀赋和发展阶段,适应人口变化趋势,尊重农村发展规律,从农业农村发展最迫切、农民反映最强烈的实际问题入手,因地制宜、注重实效,分阶段扎实稳步推进。
\end{itemize}

主要目标是:

\begin{itemize}
    \item 到2027年,农业强国建设取得明显进展。乡村全面振兴取得\textbf{实质性进展},农业农村现代化迈上新台阶。稳产保供能力巩固提升,粮食综合生产能力达到1.4万亿斤,重要农产品保持合理自给水平。农业科技装备支撑持续强化,关键核心技术攻关取得突破,育种攻关取得显著进展,农机装备补短板取得阶段性成效。现代乡村产业体系基本健全,产业链供应链价值链延伸拓展,农业国际竞争力进一步提高。宜居宜业和美乡村建设取得积极进展,农村生态环境明显改善,集中力量抓好办成一批群众可感可及的实事,逐步实现出行、用水、如厕便利,稳步提升污水治理、诊疗服务、养老保障、乡风文明、产业带农水平。脱贫攻坚成果巩固拓展,农村居民人均可支配收入增长与国内生产总值增长基本同步,城乡居民收入差距持续缩小。
    \item 到2035年,农业强国建设取得显著成效。乡村全面振兴取得\textbf{决定性进展},农业现代化基本实现,农村基本具备现代生活条件。粮食产能稳固、供给更加安全,乡村产业链升级完善、融合更加充分,乡村设施完备配套、生活更加便利,乡村公共服务普惠均等、保障更加有力,农业农村法治建设更加完善,乡村治理体系基本健全、社会更加安宁,农民收入稳定增长、城乡发展更加协调。
    \item 到本世纪中叶,\textbf{农业强国全面建成}。供给保障安全可靠,科技创新自立自强,设施装备配套完善,乡村产业链健全高效,田园乡村文明秀美,农民生活幸福美好,国际竞争优势明显,城乡全面融合,乡村全面振兴,农业农村现代化全面实现。
\end{itemize}

\subsubsection{全方位夯实粮食安全根基,把中国人的饭碗端得更牢更稳}

把保障粮食和重要农产品稳定安全供给作为头等大事,落实藏粮于地、藏粮于技战略,筑牢加快建设农业强国的物质基础。

\paragraph{(一)}全面加强耕地保护和建设。

\begin{itemize}
    \item 健全耕地数量、质量、生态“三位一体”保护制度体系,改革完善耕地占补平衡制度,各类耕地占用纳入统一管理,完善补充耕地质量验收机制,确保达到平衡标准。
    \item 守牢18.65亿亩耕地和15.46亿亩永久基本农田保护红线。分区分类开展盐碱耕地治理改良,因地制宜推动盐碱地等耕地后备资源开发。
    \item 优先把东北黑土地区、平原地区、具备水利灌溉条件地区的耕地建成高标准农田,提高建设标准和质量,推动逐步把具备条件的永久基本农田全部建成高标准农田。
    \item 完善高标准农田建设、验收、管护机制,建立健全农田建设工程质量监督检验体系。
    \item 实施耕地有机质提升行动,加强黑土地保护利用和退化耕地治理。
    \item 加强现代化灌区建设改造,健全农业水利基础设施网络。
    \item 全面提升农业防灾减灾救灾能力。
\end{itemize}

\paragraph{(二)}升粮食和重要农产品生产水平。

\begin{itemize}
    \item 扎实推进新一轮千亿斤粮食产能提升行动,深入实施粮油等主要作物大面积单产提升行动,加快建设国家粮食安全产业带。
    \item 稳定水稻、小麦生产,促进结构优化和品质提升。因地制宜发展薯类杂粮。
    \item 挖掘油菜、花生等油料作物生产潜力,拓展油茶、动物油脂等油源。
    \item 强化“菜篮子”市长负责制,发展南菜北运和冷凉地区蔬菜生产,推进生猪产业高质量发展,提升奶业竞争力,开展肉牛肉羊增量提质行动,发展现代渔业。
    \item 积极培育和推广糖料蔗良种,加快提高机械化作业水平。
    \item 推进天然橡胶老旧胶园更新改造,加快建设特种胶园。
\end{itemize}

\paragraph{(三)}健全粮食生产扶持政策。

\begin{itemize}
    \item 全面落实耕地保护和粮食安全党政同责,压实“米袋子”保供责任。
    \item 健全保障耕地用于种植基本农作物管理体系,确保粮食播种面积稳定在17.5亿亩左右、谷物面积14.5亿亩左右。
    \item 加快健全种粮农民收益保障机制,完善价格、补贴、保险等政策体系,推动粮食等重要农产品价格保持在合理水平。
    \item 完善农业保险大灾风险分散机制。
    \item 健全粮食主产区利益补偿机制,完善产粮大县奖补制度,中央预算内投资向粮食主产区倾斜,省级统筹的土地出让收入可按规定用于粮食主产区的高标准农田建设、现代种业提升等,统筹建立粮食产销区省际横向利益补偿机制。
\end{itemize}

\paragraph{(四)}强化粮食和重要农产品储备调控。

\begin{itemize}
    \item 加快完善国家储备体系。统筹推进粮食购销和储备管理体制机制改革,建立监管新模式。
    \item 健全粮食和重要农产品全链条监测预警体系。
    \item 实施饲用豆粕减量替代行动,推广低蛋白日粮技术。
    \item 健全粮食和食物节约长效机制,推进粮食播种、收获、储运、加工、消费等全链条全环节节约减损,完善反食品浪费制度。
\end{itemize}

\paragraph{(五)}构建多元化食物供给体系。

\begin{itemize}
    \item 树立大农业观、大食物观,全方位、多途径开发食物资源。在保护好生态环境的前提下,发展木本粮油。
    \item 发展优质节水高产饲草生产。
    \item 发展大水面生态渔业,推进陆基和深远海养殖渔场建设,规范发展近海捕捞和远洋渔业。
    \item 实施设施农业现代化提升行动,建设标准规范、装备先进、产出高效的现代种养设施,发展农业工厂等新形态。
    \item 发展生物科技、生物产业,壮大食用菌产业,推进合成生物产业化。
\end{itemize}

\subsubsection{全领域推进农业科技装备创新,加快实现高水平农业科技自立自强}

以农业关键核心技术攻关为引领,以产业急需为导向,加快以种业为重点的农业科技创新,推进重大农业科技突破,以发展农业新质生产力推进农业强国建设。

\paragraph{(六)}加快农业科技创新水平整体跃升。

\begin{itemize}
    \item 加强国家农业科技战略力量建设,稳定支持农业基础研究和公益科研机构,培育农业科技领军企业,构建梯次分明、分工协作、适度竞争的农业科技创新体系。
    \item 强化农业基础研究前瞻性、战略性、系统性布局,加快基因组学、预防兽医学、重大病虫害成灾和气象灾害致灾机理等研究突破。
    \item 改善农业领域国家实验室和全国重点实验室条件,建成一批世界一流农业科研机构和研究型农业高校。
    \item 加强农业科技成果转化,推动农业主产区与科技创新活跃区深度合作。
    \item 加强农业知识产权保护和侵权打击。
    \item 健全公益性和经营性相结合的农业科技推广体系,加强基层农技推广队伍建设。
\end{itemize}

\paragraph{(七)}推动种业自主创新全面突破。

\begin{itemize}
    \item 深入实施种业振兴行动,加快实现种业科技自立自强、种源自主可控。
    \item 加强种质资源保护利用,建设国际一流的国家农业种质资源保存、鉴定、创制和基因挖掘重大设施,推进种质资源交流共享。
    \item 实施育种联合攻关和畜禽遗传改良计划,加快建设南繁硅谷。
    \item 实施生物育种重大专项,选育高油高产大豆、耐盐碱作物等品种,加快生物育种产业化应用。
    \item 健全植物新品种保护制度。
    \item 加强现代化育制种基地建设,健全重大品种支撑推广体系和种源应急保障体系。
\end{itemize}

\paragraph{(八)}推进农机装备全程全面升级。

\begin{itemize}
    \item 加强大型高端智能农机、丘陵山区适用小型机械等农机装备和关键零部件研发应用,加快实现国产农机装备全面支撑农业高质高效发展。
    \item 推进老旧农机报废更新,优化农机装备结构。
    \item 打造重要农机装备产业集群,建立上下游稳定配套、工程电子等领域相关企业协同参与的产业格局。
    \item 推进农机农艺深度融合,推动农机装备研发制造、熟化定型、推广应用衔接贯通,实现种养加全链条高性能农机装备应用全覆盖。
\end{itemize}

\paragraph{(九)}促进数字技术与现代农业全面融合。

\begin{itemize}
    \item 建立健全天空地一体化农业观测网络,完善农业农村统计调查监测体系,建设全领域覆盖、多层级联通的农业农村大数据平台,健全涉农数据开发利用机制。
    \item 研发具有自主知识产权的智慧农业技术,健全智慧农业标准体系,释放农业农村数字生产力。
    \item 实施智慧农业建设工程,推动规模化农场(牧场、渔场)数字化升级,培育链条完整、协同联动的智慧农业集群。
\end{itemize}

\subsubsection{全环节完善现代农业经营体系,促进小农户和现代农业发展有机衔接}

巩固和完善农村基本经营制度,牢牢守住土地公有制性质不改变、耕地红线不突破、农民利益不受损的底线,发展农业适度规模经营,促进农业经营主体高质量发展,为加快建设农业强国增活力添动能。

\paragraph{(十)}提升家庭经营集约化水平。

\begin{itemize}
    \item 有序推进第二轮土地承包到期后再延长30年试点,健全延包配套制度。
    \item 深化承包地所有权、承包权、经营权分置改革,健全承包地集体所有权行使机制,依法保护承包农户合法土地权益,探索放活土地经营权的有效途径。
    \item 实施小农户能力提升工程,鼓励小农户通过联户经营、联耕联种等方式开展生产。
\end{itemize}

\paragraph{(十一)}推进农村集体经济活权赋能。

\begin{itemize}
    \item 深化农村集体产权制度改革,健全农村集体资产监督管理服务体系。
    \item 发挥农村集体经济组织贴近农户、服务综合、辐射带动等优势,完善农村集体经济组织在服务农业生产方面的统筹协调功能。
    \item 发展新型农村集体经济,构建产权明晰、分配合理的运行机制,赋予农民更加充分的财产权益。
\end{itemize}

\paragraph{(十二)}推进新型农业经营主体提质增效。

\begin{itemize}
    \item 引导农户发展家庭农场,提升家庭农场生产经营能力。
    \item 支持农民合作社依法自愿组建联合社,鼓励农民合作社开展农产品加工、仓储物流、市场营销等。
    \item 完善农业经营体系,完善承包地经营权流转价格形成机制,促进农民合作经营,推动新型农业经营主体扶持政策同带动农户增收挂钩。
    \item 提升农业产业化水平,健全各类市场主体联农带农利益联结机制。
    \item 深化供销合作社综合改革。
    \item 强化国有农场农业统一经营管理和服务职能,推进城乡及垦区一体化协调发展。
\end{itemize}

\paragraph{(十三)}健全便捷高效的农业社会化服务体系。

\begin{itemize}
    \item 加强农业社会化服务主体能力建设,引导各类涉农主体向社会化服务领域拓展。
    \item 创新推广单环节、多环节托管等服务模式,推动服务由粮油作物向经济作物、畜禽水产养殖等领域拓展,由产中向产前、产后环节延伸。
    \item 完善农业社会化服务标准,提升规范化服务水平。
\end{itemize}

\subsubsection{全链条推进农业产业体系升级,提升农业综合效益}

依托农业农村特色资源,开发农业多种功能,挖掘乡村多元价值,加快构建粮经饲统筹、农林牧渔并举、产加销贯通、农文旅融合的现代乡村产业体系,把农业建成现代化大产业。

\paragraph{(十四)}推动农产品加工流通优化升级。

\begin{itemize}
    \item 完善国家农产品加工技术研发体系,开发类别多样、品质优良的加工产品。
    \item 完善全国农产品流通骨干网络,优化产地冷链集配中心布局,推动有需求的县乡村加强田头冷藏保鲜设施建设。
    \item 健全县乡村物流配送体系,鼓励大型电商平台、物流、商贸等主体下沉农村,发展农村电商服务网点。
\end{itemize}

\paragraph{(十五)}推动农业优质化品牌化提升。

\begin{itemize}
    \item 深入推进农业品种培优、品质提升、品牌打造和标准化生产,增加绿色优质农产品供给。
    \item 实施农业标准化提升计划。
    \item 建立健全农产品品质评价和认证制度,完善农产品质量安全监管监测体系。
    \item 培育一批品质过硬、竞争力强的区域公用品牌、企业品牌和产品品牌。
    \item 加强中国农业品牌文化赋能,推进农业品牌与中华优秀传统文化元素相融合。
\end{itemize}

\paragraph{(十六)}加快发展乡村特色产业。

\begin{itemize}
    \item 做精做优乡村特色种养业,做好地方特色品种筛选,发展产地清洁、全程贯标、品质优良的特色种养。
    \item 创新发展乡村特色手工业,培育乡村工匠,推进乡村传统工艺振兴。
    \item 深度开发乡村特色文化产业,加强农业文化遗产、民族村寨、传统建筑等保护。
\end{itemize}

\paragraph{(十七)}促进乡村产业融合发展。

\begin{itemize}
    \item 推进生产、加工、流通等全环节升级,拓展产业增值增效空间。
    \item 深入推进优势特色产业集群、现代农业产业园、农业产业强镇建设,实施农村产业融合发展项目。
    \item 扶优培强农业产业化龙头企业,做大引领行业发展的产业链“链主”企业和区域头部企业,做优中小企业,形成生产协同、技术互补、要素共享的企业发展阵型。
    \item 大力培育乡村新产业新业态,推动农业与旅游、教育、康养等产业深度融合,丰富休闲农业和乡村旅游产品。
\end{itemize}

\paragraph{(十八)}加快推进农业全面绿色转型。

\begin{itemize}
    \item 全面推行农业用水总量控制和定额管理,开展地下水超采综合治理,深入推进农业水价综合改革。
    \item 健全耕地轮作休耕制度,加强受污染耕地治理和安全利用。
    \item 保护重要生物物种和遗传资源,加强农业生物安全管理。
    \item 健全重要农业资源台账制度。
    \item 推进农业面源污染综合防治,发展生态循环农业。
    \item 健全山水林田湖草沙一体化保护和系统治理机制。
    \item 实施好长江十年禁渔和海洋伏季休渔,推进黄河流域农业深度节水控水。
    \item 建立农业生态环境保护监测制度。
    \item 推进生态综合补偿,健全横向生态保护补偿机制,统筹推进生态环境损害赔偿。
\end{itemize}

\subsubsection{进一步深化农业对外合作,培育农业国际竞争新优势}

坚持在开放中合作、在合作中共赢,增强国内国际两个市场两种资源联动效应,着力开创更大范围、更宽领域、更深层次农业对外合作新局面。

\paragraph{(十九)}提高农业国际竞争力。

\begin{itemize}
    \item 鼓励国内优势农产品参与国际竞争,推动农产品外贸转型升级。
    \item 实施农业服务贸易促进行动。
    \item 优化农业领域外商投资准入特别管理措施,发挥自由贸易试验区、海南自由贸易港等先行先试作用。
    \item 深耕多双边农业合作关系,推进全球发展倡议粮食安全领域合作。
    \item 实施农业领域国际大科学计划和大科学工程。
    \item 深入参与全球粮农治理。
\end{itemize}

\subsubsection{高质量推进宜居宜业和美乡村建设,提升农村现代生活水平}

\paragraph{(二十)}持续提升乡村建设水平。

\begin{itemize}
    \item 强化县域国土空间规划对城镇、村庄、产业园区等空间布局的统筹,分类编制村庄规划。
    \item 深入实施乡村建设行动,逐步提高乡村基础设施完备度、公共服务便利度、人居环境舒适度。
    \item 推进乡村基础设施提档升级,加强农村交通运输网、供水设施、能源体系和新型基础设施建设。
    \item 完善农村公共服务体系,提高农村教育质量,办好必要的乡村小规模学校,促进乡村医疗卫生体系健康发展,提升农村养老服务水平,健全农村老年人、留守妇女儿童和残疾人关爱服务体系。
    \item 扎实推进农村厕所革命,分区分类推进农村生活污水垃圾治理,健全农村人居环境长效管护机制。
    \item 编制村容村貌提升导则,保护乡村特色风貌。
\end{itemize}

\paragraph{(二十一)}整体提升乡村治理效能。

\begin{itemize}
    \item 坚持和发展新时代“枫桥经验”,健全党组织领导的自治、法治、德治相结合的乡村治理体系。
    \item 深入推进抓党建促乡村全面振兴,建好建强农村基层党组织,健全以财政投入为主的稳定的村级组织运转经费保障制度。
    \item 培养选拔村党组织带头人。
    \item 推动基层纪检监察组织和村务监督委员会有效衔接。
    \item 推进村民自治组织规范化建设,加强基层群团组织建设,推动乡村服务性、公益性、互助性社会组织健康发展。
    \item 健全村民会议、村民代表会议等制度,完善农村基层公共法律服务体系。
    \item 健全农村扫黑除恶常态化机制,依法管理农村宗教事务。
    \item 促进治理和服务重心向基层下移,健全乡镇(街道)职责和权力、资源相匹配制度,加强乡镇(街道)服务管理力量,推广务实管用的治理方式。
    \item 加强县乡村应急管理和消防安全体系建设。
\end{itemize}

\paragraph{(二十二)}深化农村精神文明建设。

\begin{itemize}
    \item 弘扬和践行社会主义核心价值观,拓展新时代文明实践中心建设,培养新时代农民。
    \item 实施文明乡风建设工程,强化道德教化和激励约束,持续推进农村移风易俗。
    \item 加强传统村落保护传承。
    \item 推进中国传统节日振兴,以农民为主体办好中国农民丰收节。
    \item 实施农耕文化传承保护工程,强化农业文化遗产挖掘整理和保护利用。
    \item 加强农村公共文化服务体系建设,实施文化惠民工程。
    \item 完善农村群众文艺扶持机制,增加富有农耕农趣农味、体现和谐和顺和美的乡村文化产品服务供给。
\end{itemize}

\subsubsection{促进城乡融合发展,缩小城乡差别}

\paragraph{(二十三)}拓宽农民增收致富渠道。

\begin{itemize}
    \item 壮大县域富民产业,推动县域产业加快融入邻近大中城市产业链供应链创新链,引导劳动密集型产业梯次向县域转移。
    \item 支持涉农高校、企业办好高质量职业技能培训,鼓励开展农民工急需紧缺职业专项培训,健全跨区域就业服务机制,建立区域劳务协作平台,促进农民就业拓岗增收。
    \item 加强农村宅基地规范管理,允许农户合法拥有的住房通过出租、入股、合作等方式盘活利用。
    \item 有序推进农村集体经营性建设用地入市改革,健全土地增值收益分配机制。
    \item 健全城乡居民基本医疗保险筹资机制,推进基本养老保险制度覆盖全体农村居民并适时提高基础养老金标准,健全低保标准制定和动态调整机制。
\end{itemize}

\paragraph{(二十四)}推进城乡融合发展。

\begin{itemize}
    \item 统筹新型工业化、新型城镇化和乡村全面振兴,全面提高城乡规划、建设、治理融合水平。
    \item 完善城乡融合发展体制机制和政策体系,促进城乡要素平等交换、双向流动,推动公共资源均衡配置。
    \item 健全政府投资与金融、社会投入联动机制,稳步提高土地出让收入用于农业农村比例。
    \item 构建城乡统一的建设用地市场,新编县乡级国土空间规划安排不少于10%建设用地指标用于农业农村。
    \item 健全农业转移人口市民化配套政策体系,推行由常住地登记户口提供基本公共服务制度,推动符合条件的农业转移人口社会保险、住房保障、随迁子女义务教育等享有同迁入地户籍人口同等权利。
    \item 把县域作为城乡融合发展的重要切入点,推动城乡交通道路连接、供电网络互联、客运物流一体,促进城乡基本公共服务标准统一、制度并轨。
    \item 加强县域商业体系建设,发掘农村消费潜力。
\end{itemize}

\paragraph{(二十五)}增强脱贫地区和脱贫群众内生发展动力。

\begin{itemize}
    \item 完善覆盖农村人口的常态化防止返贫致贫机制,建立农村低收入人口和欠发达地区分层分类帮扶制度。
    \item 按照巩固、升级、盘活、调整的原则,推进脱贫地区帮扶产业高质量发展,构建成长性好、带动力强的帮扶产业体系,将发展联农带农富农产业作为中央财政衔接推进乡村全面振兴补助资金优先支持内容。
    \item 加快补齐脱贫地区农村基础设施短板,优先布局产业发展所需配套设施。
    \item 发展面向脱贫地区的职业教育,持续开展“雨露计划+”就业促进行动。
    \item 持续做好中央单位定点帮扶,深化东西部协作,推进“万企兴万村”行动。
    \item 集中支持国家乡村振兴重点帮扶县,强化易地扶贫搬迁后续扶持。
    \item 深入实施兴边富民行动。
    \item 健全脱贫攻坚国家投入形成资产的长效管理机制。
\end{itemize}

\subsubsection{保障措施}

加快建设农业强国必须坚持和加强党的全面领导,把党的领导贯彻到农业强国建设各领域各方面各环节,确保党中央决策部署落到实处。

\begin{itemize}
    \item 各级党委和政府要高度重视“三农”工作,坚决扛起主体责任,将农业强国建设列入重要议事日程,纳入国民经济和社会发展规划等,确保农业强国建设各项任务不断取得实质性进展。
    \item 各有关部门要加强政策协同配合,强化规划、项目、资金、要素间的有效衔接,避免重复安排建设。
    \item 对本规划确定的重大任务和重大工程项目,优先保障土地供应。
    \item 完善乡村振兴投入机制,健全农村金融服务体系,发展多层次农业保险,重点支持农业强国建设关键领域和薄弱环节。
    \item 加强涉农干部培训,提高“三农”工作本领,打造一支政治过硬、适应新时代要求、具有服务农业强国建设能力的“三农”干部队伍。
    \item 实施乡村振兴人才支持计划,坚持本土培养和外部引进相结合,加强高校涉农专业建设,提升职业教育水平,建强农业强国建设人才队伍。
    \item 建立健全农业农村法律规范体系。
    \item 推动东部沿海发达地区有条件省份率先建成农业强省,鼓励中西部地区经济发展水平较高、资源条件较好的市地加快建设农业强市,引导有条件的县(市、区)加快建设农业强县,分类探索差异化、特色化发展模式。
    \item 重大事项及时按程序向党中央、国务院请示报告。
\end{itemize}

